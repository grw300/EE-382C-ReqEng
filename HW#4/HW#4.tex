\documentclass{article}
\usepackage[margin=3.5cm]{geometry}
\usepackage{pdfpages}
\usepackage{graphicx}
\usepackage{verbatim}
\DeclareGraphicsExtensions{.pdf,.png,.jpg}	

\definecolor{codegray}{gray}{0.9}
\newcommand{\code}[1]{\colorbox{codegray}{\texttt{#1}}}

\title{Homework III}
\author{Gregory Williams\\GW4975\\EE 382C Requirements Engineering}
\date{11/06/2015}

\begin{document}
	\maketitle
	
	\section*{4.1.1}
	The invariant is violated by Fred who has an ID with length equal to 5. The invariant \code{StudentIdMustBeLength4} works within the \code{University} context; it loads all of the \code{students} that belong to (i.e. are enrolled) at the \code{University} and checks each student's \code{id} to see if the length is equal to \code{4}.\\
	\begin{center}
	\parbox{0.5\textwidth}{
		\code{context University inv StudentIdMustBeLength4:\\
    	    self.students->forAll(s : Student | (s.id.size = 4))}
	}
	\end{center}
	
	\section*{4.1.2}
	
	(invariant code)
-- OCL constraints

constraints

context University
    -- A student's id number must be exactly
    -- four characters long
    inv StudentIdMustBeLength4:
       self.students->forAll(s|s.id.size() = 4)

	-- A student's id number must be unique
	inv StudentIdMustBeUnique:
		self.students->forAll( s1, s2 | s1.id = s2.id implies s1 = s2 )
	
	\section*{4.1.3}
	(invariant result)
	checking invariant (2) `University::StudentIdMustBeUnique': FAILED.
  -> false : Boolean
Results of subexpressions:
  University.allInstances : Set(University) = Set{@ut}
  self : University = @ut
  self.students : Set(Student) = Set{@fred,@sam,@sue}
  s1 : Student = @fred
  s1.id : String = '56789'
  s2 : Student = @fred
  s2.id : String = '56789'
  (s1.id = s2.id) : Boolean = true
  s1 : Student = @fred
  s2 : Student = @fred
  (s1 = s2) : Boolean = true
  ((s1.id = s2.id) implies (s1 = s2)) : Boolean = true
  s1 : Student = @fred
  s1.id : String = '56789'
  s2 : Student = @sam
  s2.id : String = '1234'
  (s1.id = s2.id) : Boolean = false
  ((s1.id = s2.id) implies (s1 = s2)) : Boolean = true
  s1 : Student = @fred
  s1.id : String = '56789'
  s2 : Student = @sue
  s2.id : String = '1234'
  (s1.id = s2.id) : Boolean = false
  ((s1.id = s2.id) implies (s1 = s2)) : Boolean = true
  s1 : Student = @sam
  s1.id : String = '1234'
  s2 : Student = @fred
  s2.id : String = '56789'
  (s1.id = s2.id) : Boolean = false
  ((s1.id = s2.id) implies (s1 = s2)) : Boolean = true
  s1 : Student = @sam
  s1.id : String = '1234'
  s2 : Student = @sam
  s2.id : String = '1234'
  (s1.id = s2.id) : Boolean = true
  s1 : Student = @sam
  s2 : Student = @sam
  (s1 = s2) : Boolean = true
  ((s1.id = s2.id) implies (s1 = s2)) : Boolean = true
  s1 : Student = @sam
  s1.id : String = '1234'
  s2 : Student = @sue
  s2.id : String = '1234'
  (s1.id = s2.id) : Boolean = true
  s1 : Student = @sam
  s2 : Student = @sue
  (s1 = s2) : Boolean = false
  ((s1.id = s2.id) implies (s1 = s2)) : Boolean = false
  self.students->forAll(s1, s2 : Student | ((s1.id = s2.id) implies (s1 = s2))) : Boolean = false
  University.allInstances->forAll(self : University | self.students->forAll(s1, s2 : Student | ((s1.id = s2.id) implies (s1 = s2)))) : Boolean = false

\section*{4.1.4}
  -- OCL constraints
constraints

context University
    -- A student may be a GraduateStudent
	-- or an UndergraduateStudent
    -- but not both
    inv StudentIsGradOrUndergradNotBoth:
       self.undergraduates->intersection(self.graduates)->isEmpty()
  \section*{4.1.5}
  checking invariant (1) `University::StudentIsGradOrUndergradNotBoth': FAILED.
  -> false : Boolean
Results of subexpressions:
  University.allInstances : Set(University) = Set{@ut}
  self : University = @ut
  self.undergraduates : Set(Student) = Set{@sam}
  self : University = @ut
  self.graduates : Set(Student) = Set{@sam}
  self.undergraduates->intersection(self.graduates) : Set(Student) = Set{@sam}
  self.undergraduates->intersection(self.graduates)->isEmpty : Boolean = false
  University.allInstances->forAll(self : University | self.undergraduates->intersection(self.graduates)->isEmpty) : Boolean = false

 \section*{4.1.6}
-- OCL constraints
constraints

context University
    -- A student may not exceed the maxApprovedSemesterHours
	-- All Courses offered are assumed to have 3 credit hours
    inv StudentIsGradOrUndergradNotBoth:
       self.students->forAll(s | s.takingCourses->size() * 3 <= s.maxApprovedSemesterHours)

\section*{4.1.7}
 checking invariant (1) `University::StudentIsGradOrUndergradNotBoth': FAILED.
  -> false : Boolean
Results of subexpressions:
  University.allInstances : Set(University) = Set{@ut}
  self : University = @ut
  self.students : Set(Student) = Set{@sam}
  s : Student = @sam
  s.takingCourses : Set(Course) = Set{@BUS311,@CS306,@E306,@EE302,@EE323,@EE338,@EE379K}
  s.takingCourses->size : Integer = 7
  3 : Integer = 3
  (s.takingCourses->size * 3) : Integer = 21
  s : Student = @sam
  s.maxApprovedSemesterHours : Integer = 18
  ((s.takingCourses->size * 3) <= s.maxApprovedSemesterHours) : Boolean = false
  self.students->forAll(s : Student | ((s.takingCourses->size * 3) <= s.maxApprovedSemesterHours)) : Boolean = false
  University.allInstances->forAll(self : University | self.students->forAll(s : Student | ((s.takingCourses->size * 3) <= s.maxApprovedSemesterHours))) : Boolean = false
 
\section*{4.2.1}

-- OCL constraints
constraints

context Student
	inv studentEnrolledInUniversity: self.isEnrolledAt.students->includes(self)

context Student :: drop(c : Course)
	pre studentIsRegistered: self.takingCourses->includes(c)
	pre studentHasMoreThanOneClass: self.takingCourses->size() > 1
	post studentIsNotRegistered: not self.takingCourses->includes(c)
	post studentDidNotDropOtherCoursesRegistered: self.takingCourses->including(c) = self.takingCourses@pre
	post droppedCourseNotFull: c.isFull = false
	post studentStillEnrolledInUniversity: self.isEnrolledAt = self.isEnrolledAt@pre
	post onlyThisStudentWasRemoved: c.studentsEnrolled->including(self) = c.studentsEnrolled@pre
	
 \section*{4.2.2}
 
 !create ut : University
!create sam : Student
!create sue : Student
!insert (sam,ut) into EnrolledAtUniversity
!insert (sue,ut) into EnrolledAtUniversity
!create EE302 : Course
!create CS306 : Course
!create BUS311 : Course
!create EE411 : Course
!create EE379K : Course
!create E306 : Course
!create EE338 : Course
!create EE323 : Course
!insert (sam,EE302) into TakingCourse
!insert (sam,CS306) into TakingCourse
!insert (sam,BUS311) into TakingCourse
!insert (sam,EE323) into TakingCourse
!insert (sam,EE379K) into TakingCourse
!insert (sam,E306) into TakingCourse
!insert (sam,EE338) into TakingCourse
!insert (sue,EE302) into TakingCourse
!insert (sue,CS306) into TakingCourse
!insert (sue,BUS311) into TakingCourse
!insert (sue,EE323) into TakingCourse
!set EE302.isFull := true
!openter sam drop(EE302)
!delete (sam,EE302) from TakingCourse
!set EE302.isFull := false
!opexit

\section*{4.2.3}

CR2.cmd> !openter sam drop(EE302)
precondition `studentIsRegistered' is true
precondition `studentHasMoreThanOneClass' is true
CR2.cmd> !delete (sam,EE302) from TakingCourse
CR2.cmd> !set EE302.isFull := false
CR2.cmd> !opexit
postcondition `studentIsNotRegistered' is true
postcondition `studentDidNotDropOtherCoursesRegistered' is true
postcondition `droppedCourseNotFull' is true
postcondition `studentEnrolledInUniversity' is true
postcondition `onlyThisStudentWasRemoved' is true
 \end{document}