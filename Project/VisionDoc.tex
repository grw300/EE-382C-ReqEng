\documentclass[12pt]{article}

% Any percent sign marks a comment to the end of the line

% Every latex document starts with a documentclass declaration like this
% The option dvips allows for graphics, 12pt is the font size, and article
%   is the style

\usepackage[pdftex]{graphicx}
\usepackage{url}
\usepackage{xcolor}
\usepackage{sectsty}

\subsectionfont{\color{blue}}  % sets colour of chapters
\sectionfont{\color{cyan}}

% These are additional packages for "pdflatex", graphics, and to include
% hyperlinks inside a document.

\setlength{\oddsidemargin}{0.25in}
\setlength{\textwidth}{6.5in}
\setlength{\topmargin}{0in}
\setlength{\textheight}{8.5in}

% These force using more of the margins that is the default style

\begin{document}

% Everything after this becomes content
% Replace the text between curly brackets with your own
\begin{titlepage}
\title{\color{cyan}Electronic Health Records Managment}
\author{for\\{\color{blue}Independent Health Care Service Providers}\\
	\\{\small Requirements Engineers:}
\\Robert Pate\\Neel Shah\\Adam Whipple\\Gregory Williams}
\date{\today}

% You can leave out "date" and it will be added automatically for today
% You can change the "\today" date to any text you like


\maketitle
\vfill
\end{titlepage}
% This command causes the title to be created in the document

\section{Product Overview}

We envision the creating of a Electronic Health Records Management System that empowers independent health care providers to effectively create, maintain and provide access to patient records. The System will push the ownership of medical records towards the patient. We see the ability to provide enhanced security that meets and excess Health Insurance Portability and Accountability Act (HIPAA) requirements without burdening the providers with excess legalese. We see the system as a way for providers to give patients ownership over their health records and inspire patient loyalty with easy-to-use public facing interfaces for patient applications (mobile applications/website access);

\section{Stakeholders}

\subsection{Domain Viewpoint Hierarchy}

\subsection{Description of Stakeholders}

\subsection{Priority Stakeholders}
\begin{itemize}
\item Independent Health Care Providers
\\System main customers
\item Patients
\\Systems design goal to give patients ownership over medical records.
\item Health Care Provider Office Personal
\\First point of contact with The System.
\end{itemize}

\section{Operational Reference Model}

Assuming that the best telescope for your work is one of the two 0.5 meters
(CDK20N at Moore Observatory, CDK20S at Mt. Kent), you will have a choice of
filters:  Sloan filter set (g, r, i, or z),  Johnson-Cousins (U, B, V, R, or I),
color imaging (B, G, R, or clear), and narrow band (S $[II]$, red continuum,
H$\alpha$,  O$[III]$.  Identify which filters are of interest.

A typical exposure time for a magnitude 12 star to about half saturation is 100
seconds, but it depends on the filter choice.  Based on this, estimate how many
exposures you will need, and what total time you require.  In some cases, for
example studying an eclipsing or variable star, or an exoplanet transit, you
would use only one filter and make many measurements over a night.  In others,
you may make only a few exposures in each filter, and try many different
filters.   Changing filter sets takes an operator and several minutes, but
changing filters within one set (e.g. a different Sloan filter) takes only a few
seconds.

We have other telescopes that may be available at Moore Observatory this season.
There is a wide field astrograph that has a field of view of $4^\circ$ and is a
fast $f/4$,  especially good for large nebula, comets, or surveys.  A 14-inch
(0.36 meter) Celestron  telescope can be equipped with a fast camera for
planetary imaging.  A 27-inch (0.7 meter)  corrected Dall-Kirkham is scheduled
to be be delivered to Australia this fall, although we are unsure of the actual
date it could see light yet.  


 
\begin{thebibliography}{99}

\bibitem{gonzalez2012} Jonay I. Gonz\'{a}lez Hern\'{a}ndez, 
Pilar Ruiz-Lapuente,	
Hugo M. Tabernero,	
David Montes,	
Ramon Canal,	
Javier M\'{e}ndez	
and Luigi R. Bedin,
{No surviving evolved companions of the progenitor of SN1006},
Nature, {\bf 489}, 533-536 (2012).

\end{thebibliography}



\end{document}