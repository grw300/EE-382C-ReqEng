% Flowcharting techniques for easy maintenance
% Author: Brent Longborough
\documentclass[x11names]{article}
\usepackage{tikz}
\usetikzlibrary{shapes,arrows,chains,positioning}
%%%<
\usepackage{verbatim}
\usepackage[active,tightpage]{preview}
\PreviewEnvironment{tikzpicture}
\setlength\PreviewBorder{5mm}%
%%%>
\begin{comment}
:Title: Easy-maintenance flowchart
:Tags: flowcharts
:Author: Brent Longborough
:Slug: flexible-flow-chart

  This TikZ example illustrates a number of techniques for making TikZ
  flowcharts easier to maintain:
    * Use of <on chain> and <on grid> to simplify positioning
    * Use of global <node distance> options to eliminate the need to 
      specify individual inter-node distances
    * Use of <join> to reduce the need for references to node names
    * Use of <join by> styles to tailor specific connectors
    * Use of <coordinate> nodes to provide consistent layout for
      parallel flow lines
    * A method for consistent annotation of decision box exits
    * A technique for marking coordinate nodes (for layout debugging)

    I encourage you to tinker at this file - add intermediate boxes,
    alter the global distance settings, and so on, to see how well (or
    ill!) it adapts.
\end{comment}
\begin{document}
% =================================================
% Set up a few colours
\colorlet{lcfree}{Green3}
\colorlet{lcnorm}{Blue3}
\colorlet{lccong}{Red3}
% -------------------------------------------------
% Set up a new layer for the debugging marks, and make sure it is on
% top
\pgfdeclarelayer{marx}
\pgfsetlayers{main,marx}
% A macro for marking coordinates (specific to the coordinate naming
% scheme used here). Swap the following 2 definitions to deactivate
% marks.
\providecommand{\cmark}[2][]{%
  \begin{pgfonlayer}{marx}
    \node [nmark] at (c#2#1) {#2};
  \end{pgfonlayer}{marx}
  } 
\providecommand{\cmark}[2][]{\relax} 
% -------------------------------------------------
% Start the picture
\begin{tikzpicture}[%
    >=triangle 60,              % Nice arrows; your taste may be different
    start chain=going below,    % General flow is top-to-bottom
    node distance=6mm and 60mm, % Global setup of box spacing
    every join/.style={norm},   % Default linetype for connecting boxes
    ]
% ------------------------------------------------- 
% A few box styles 
% <on chain> *and* <on grid> reduce the need for manual relative
% positioning of nodes
\tikzset{
  base/.style={draw, on chain, on grid, align=center, minimum height=4ex},
  proc/.style={base, rectangle, text width=8em},
  test/.style={base, diamond, aspect=2, text width=5em},
  term/.style={proc, rounded corners},
  % coord node style is used for placing corners of connecting lines
  coord/.style={coordinate, on chain, on grid, node distance=6mm and 25mm},
  % nmark node style is used for coordinate debugging marks
  nmark/.style={draw, cyan, circle, font={\sffamily\bfseries}},
  % -------------------------------------------------
  % Connector line styles for different parts of the diagram
  norm/.style={->, draw, lcnorm},
  free/.style={->, draw, lcfree},
  cong/.style={->, draw, lccong},
  it/.style={font={\small\itshape}}
}
% -------------------------------------------------
% Start by placing the nodes

% Use join to connect a node to the previous one 
\node [term] (te1) {Course Info};
\node [proc, join]       {Create Survey};
\node [proc, join] (p1)  {Dispatch Survey};

\node [proc, densely dotted, it, above right=2cm and 3cm of te1] (p0) {New Semester};

\node [term, right=of te1] (te2) {UTEIDs (Hashed)};
\node [proc, join] {Student Record};
\node [proc, join] (p2){Student Names};

\node [test, below left=4cm and 3cm of p2] (t1) {All students?};
\node [proc, join] (p3){Collected Preferences};
\node [proc, join] (p4){Audit};
\node [test, join] (t2) {Conflicts?};
\node [term, join] (te3){Registration List};

\node [proc, right=4cm of t2] (p5){Notify Admin};

% -------------------------------------------------
% A couple of boxes have annotations
\node [above=0mm of te1, it] {(UT Data)};
\node [above=0mm of te2, it] {(FERPA Concerns)};
% -------------------------------------------------

\draw [->,lcnorm] (p1.south)  |- (t1.west);
\draw [->,lcnorm] (p2.south)  |- (t1.east);
\draw [<-*,lcnorm] (p1.east)	  -| (t1.north);
\draw [-,lcnorm] (p2.west) -- (p1.east) node [midway, above, it, black] {Unique per student};
\path (t1.south) to node [near start, xshift=1em] {$y$} (p3); 
\path (t1.south) to node [near start, xshift=1em, yshift=2.5em] {$n$} (p1);

\draw [*->,lcnorm] (t2.east) -- (p5.west);
\draw [->,lcnorm] (p5.north) |- (p4.east);
\path (t2.east) to node [near start, yshift=.75em] {$y$} (p5); 
\path (t2.south) to node [near start, xshift=1em] {$n$} (te3);

% -------------------------------------------------
\end{tikzpicture}
% =================================================
\end{document}