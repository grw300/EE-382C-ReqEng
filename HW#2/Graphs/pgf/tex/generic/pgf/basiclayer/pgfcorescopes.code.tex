% Copyright 2006 by Till Tantau
%
% This file may be distributed and/or modified
%
% 1. under the LaTeX Project Public License and/or
% 2. under the GNU Public License.
%
% See the file doc/generic/pgf/licenses/LICENSE for more details.

\ProvidesFileRCS $Header: /cvsroot/pgf/pgf/generic/pgf/basiclayer/pgfcorescopes.code.tex,v 1.44 2013/10/09 18:30:27 tantau Exp $

% Globals

\newbox\pgfpic
\newbox\pgf@hbox

\newbox\pgf@layerbox@main

\newcount\pgf@picture@serial@count



% This if decides whether the position of pictures on the page is
% protocolled or not. Normally,
% this is switched off as it works only with certain drivers and it
% causes external files to be written. When switched on, the position
% of pgfpictures are protocolled and can be referenced using
% \pgfsys@getposition{XXX} where XXX is the value of \pgfpictureid
% inside the picture.
\newif\ifpgfrememberpicturepositiononpage




% Scopes


% Pgf scope environment. All changes of the graphic state are local to
% the scope.
%  
% Example:
%
% \begin{pgfscope}
%    \pgfsetlinewidth{3pt}
%    \pgfline{\pgfxy(0,0)}{\pgfxy(3,3)}
% \end{pgfscope}

\def\pgfscope{%
  \pgfsyssoftpath@setcurrentpath\pgfutil@empty%
  \pgfsys@beginscope%
    \pgf@resetpathsizes%
    \edef\pgfscope@linewidth{\the\pgflinewidth}%
    \expandafter\let\expandafter\pgfscope@stroke@color\csname\string\color@pgfstrokecolor\endcsname%
    \expandafter\let\expandafter\pgfscope@fill@color\csname\string\color@pgffillcolor\endcsname%
    \begingroup}
\def\endpgfscope{%
    \endgroup%
    \global\pgflinewidth=\pgfscope@linewidth%
    \expandafter\global\expandafter\let\csname\string\color@pgfstrokecolor\endcsname\pgfscope@stroke@color%
    \expandafter\global\expandafter\let\csname\string\color@pgffillcolor\endcsname\pgfscope@fill@color%
  \pgfsys@endscope}




% Quickly insert a box can contain normal TeX text at the origin.
% 
% #1 = box of width/height and depth 0pt
% 
% Example:
%
% \pgfqbox{\mybox}

\def\pgfqbox#1{%
  \pgfsys@hbox#1%
} 


% Insert a box that can contain normal TeX text at the origin, but
% with the current coordinate transformation matrix synced with the
% low-level transformation matrix.
% 
% #1 = box of width/height and depth 0pt
%
% In essence, this command does the same as if you first said
% \pgflowlevelsynccm and then \pgfqbox. However, pgf will use a
% ``TeX-translation'' for the translation part of the transformation
% cm. This will ensure that hyperlinks ``survive'' at least
% translations. 
% 
% Example:
%
% \pgfqboxsynced{\mybox}

\def\pgfqboxsynced#1{%
  \pgfsys@hboxsynced#1%
}


% Puts some text in a box and inserts it with the current
% transformations applied.
% 
% #1 = List of optional positioning. Possible values are ``left'', ``right'',
%      ``top'', ``bottom'' and ``base''.
% #2 = TeX text. May contain verbatims.
% 
% Example:
%
% \pgftransformshift{\pgfpoint{1cm}{0cm}}
% \pgftext{Hello World!}

\def\pgftext{\pgfutil@ifnextchar[\pgf@text{\pgf@text[]}}%
\def\pgf@text[#1]{%
  \def\pgf@text@options{#1}%
  \pgf@maketext\pgf@after@text}
\def\pgf@after@text{%
  {%
  \def\pgf@text@hshift{center}%
  \def\pgf@text@vshift{center}%
  \expandafter\pgf@text@setkeys\expandafter{\pgf@text@options}%
  \csname pgf@halign\pgf@text@hshift\endcsname%
  \csname pgf@valign\pgf@text@vshift\endcsname%
  %
  \pgfapproximatenonlineartransformation%
  % Protocol sizes:
  \pgf@process{\pgfpointtransformed{\pgfqpoint{0pt}{\dp\pgf@hbox}}}%
  \pgf@protocolsizes{\pgf@x}{\pgf@y}%
  \pgf@process{\pgfpointtransformed{\pgfqpoint{\wd\pgf@hbox}{\dp\pgf@hbox}}}%
  \pgf@protocolsizes{\pgf@x}{\pgf@y}%
  \pgf@process{\pgfpointtransformed{\pgfqpoint{0pt}{\ht\pgf@hbox}}}%
  \pgf@protocolsizes{\pgf@x}{\pgf@y}%
  \pgf@process{\pgfpointtransformed{\pgfqpoint{\wd\pgf@hbox}{\ht\pgf@hbox}}}%
  \pgf@protocolsizes{\pgf@x}{\pgf@y}%
  \pgfqboxsynced{\pgf@hbox}%
  }%
}
\def\pgf@text@setkeys{\pgfqkeys{/pgf/text}}%

\pgfkeys{/pgf/text/left/.code=\def\pgf@text@hshift{left}}
\pgfkeys{/pgf/text/center/.code=}
\pgfkeys{/pgf/text/right/.code=\def\pgf@text@hshift{right}}
\pgfkeys{/pgf/text/top/.code=\def\pgf@text@vshift{top}}
\pgfkeys{/pgf/text/bottom/.code=\def\pgf@text@vshift{bottom}}
\pgfkeys{/pgf/text/base/.code=\def\pgf@text@vshift{base}}
\pgfkeys{/pgf/text/at/.cd,.code=\pgftransformshift{#1},.value required}
\pgfkeys{/pgf/text/x/.cd,.code=\pgftransformxshift{#1},.value required}
\pgfkeys{/pgf/text/y/.cd,.code=\pgftransformyshift{#1},.value required}
\pgfkeys{/pgf/text/rotate/.cd,.code=\pgftransformrotate{#1},.value required}

\def\pgf@halignleft{}% do nothing
\def\pgf@haligncenter{\pgftransformxshift{+-.5\wd\pgf@hbox}}
\def\pgf@halignright{\pgftransformxshift{+-\wd\pgf@hbox}}%
\def\pgf@valignbase{}% do nothing
\def\pgf@valignbottom{\pgftransformyshift{+\dp\pgf@hbox}}%
\def\pgf@valigncenter{\pgftransformyshift{+.5\dp\pgf@hbox}\pgftransformyshift{+-.5\ht\pgf@hbox}}%
\def\pgf@valigntop{\pgftransformyshift{+-\ht\pgf@hbox}}%


% Internal function for creating a hbox.
\def\pgf@maketext#1{%
  \def\pgf@@maketextafter{#1}%
  \setbox\pgf@hbox=\hbox\bgroup%
    \pgfinterruptpicture%
      \bgroup%
        \aftergroup\pgf@collectresetcolor%
        \let\next=%
}
\def\pgf@collectresetcolor{%
  \pgfutil@ifnextchar\reset@color%
  {\reset@color\afterassignment\pgf@collectresetcolor\let\pgf@temp=}%
  {\pgf@textdone}%
}
\def\pgf@textdone{%
    \endpgfinterruptpicture%
  \egroup%
  \pgf@@maketextafter%  
}

\long\def\pgf@makehbox#1{%
  \setbox\pgf@hbox=\hbox{{%
    \pgfinterruptpicture%
      #1%
    \endpgfinterruptpicture%
    }}}



% Picture environment
%
% Example:
%
% \begin{pgfpicture}
%   \pgfsetendarrow{\pgfarrowto}
%   \pgfpathmoveto{\pgfpointxy{-0.9}{0.2}}
%   \pgfpathlineto{\pgfpointxy{0.9}{0.4}}
%   \pgfusepath{stroke}
% \end{pgfpicture}

\def\pgfresetboundingbox{%
    \global\pgf@picmaxx=-16000pt\relax%
    \global\pgf@picminx=16000pt\relax%
    \global\pgf@picmaxy=-16000pt\relax%
    \global\pgf@picminy=16000pt\relax%
}%

\def\pgfpicture{%
  \begingroup%
    \pgfpicturetrue%
    \global\advance\pgf@picture@serial@count by1\relax%
    \edef\pgfpictureid{pgfid\the\pgf@picture@serial@count}%
    \let\pgf@nodecallback=\pgfutil@gobble%
    \pgf@picmaxx=-16000pt\relax%
    \pgf@picminx=16000pt\relax%
    \pgf@picmaxy=-16000pt\relax%
    \pgf@picminy=16000pt\relax%
    \pgf@relevantforpicturesizetrue%
    \pgf@resetpathsizes%
    \pgfutil@ifnextchar\bgroup\pgf@oldpicture\pgf@picture}
\def\pgf@oldpicture#1#2#3#4{%
  \pgfmathsetlength\pgf@picminx{#1}%
  \pgfmathsetlength\pgf@picminy{#2}%
  \pgfmathsetlength\pgf@picmaxx{#3}%
  \pgfmathsetlength\pgf@picmaxy{#4}%
  \pgf@relevantforpicturesizefalse%
  \pgf@picture}

\def\pgf@picture{%
  \setbox\pgfpic\hbox to0pt\bgroup%
    \begingroup%
    \pgfsys@beginpicture%
      \pgfsys@beginscope%
        \begingroup%
        \pgfsetcolor{.}%
        \pgfsetlinewidth{0.4pt}%
        \pgftransformreset%
        \pgfsyssoftpath@setcurrentpath\pgfutil@empty%
        \begingroup%
          \let\pgf@setlengthorig=\setlength%
          \let\pgf@addtolengthorig=\addtolength%
          \let\pgf@selectfontorig=\selectfont%
          \let\setlength=\pgf@setlength%
          \let\addtolength=\pgf@addtolength%
          \let\selectfont=\pgf@selectfont%
          \nullfont\spaceskip0pt\xspaceskip0pt%
          \setbox\pgf@layerbox@main\hbox to0pt\bgroup%
            \begingroup%
  }
\def\endpgfpicture{%
              \ifpgfrememberpicturepositiononpage%
                \hbox to0pt{\pgfsys@markposition{\pgfpictureid}}%
              \fi%    
              % ok, now let's position the box
              \ifdim\pgf@picmaxx=-16000pt\relax%
                % empty picture. make size 0.  
                \global\pgf@picmaxx=0pt\relax%
                \global\pgf@picminx=0pt\relax%
                \global\pgf@picmaxy=0pt\relax%
                \global\pgf@picminy=0pt\relax%
              \fi%
              % Shift baseline outside:
              \pgf@relevantforpicturesizefalse%
              \pgf@process{\pgf@baseline}%
              \xdef\pgf@shift@baseline{\the\pgf@y}%
              % 
              \pgf@process{\pgf@trimleft}%
              \global\advance\pgf@x by-\pgf@picminx
              % prepare \hskip\pgf@trimleft@final.
              % note that \pgf@trimleft@final is also queried
              % by the pgf image externalization.
              \xdef\pgf@trimleft@final{-\the\pgf@x}%
              % 
              \pgf@process{\pgf@trimright}%
              \global\advance\pgf@x by-\pgf@picmaxx
              % prepare \hskip\pgf@trimright@final.
              % note that \pgf@trimright@final is also queried
              % by the pgf image externalization.
              \xdef\pgf@trimright@final{\the\pgf@x}%
        %
        \pgf@remember@layerlist@globally
            \endgroup%
            \hss%
          \egroup%
      \pgf@restore@layerlist@from@global
          \pgf@insertlayers%
        \endgroup%    
        \pgfsys@discardpath%
        \endgroup%
      \pgfsys@endscope%
    \pgfsys@endpicture%
    \endgroup%
    \hss
  \egroup%
  \pgfsys@typesetpicturebox\pgfpic%
  \endgroup%
}

\def\pgf@remember@layerlist@globally{%
  \global\let\pgf@layerlist@=\pgf@layerlist
}%
\def\pgf@restore@layerlist@from@global{%
  \let\pgf@layerlist=\pgf@layerlist@
}%
\def\pgf@insertlayers{%
  \box\pgf@layerbox@main%
}

\def\pgf@selectfont{\pgf@selectfontorig\nullfont}

\def\pgf@setlength#1#2{% these will be used only when \nullfont is active
  \begingroup% keep font setting local
    \pgfutil@selectfont% restore font
    \pgf@setlengthorig#1{#2}% calculate dimension (possibly using calc)
    \expandafter%
  \endgroup%
  \expandafter#1\expandafter=\the#1\relax}
\def\pgf@addtolength#1#2{%
  \begingroup% keep font setting local
    \pgfutil@selectfont% restore font
    \pgf@addtolengthorig#1{#2}% calculate dimension (possibly using calc)
    \expandafter%
  \endgroup%
  \expandafter#1\expandafter=\the#1\relax}


% Sets the baseline at the y-coordinate of a given point
%
% #1 = point
%
% Sets the baseline of the picture to the y-coordinate of a given
% point. However, the point will be evaluated *at the end of the
% picture*. 
%
% Example:
%
% \pgfsetbaselinepointlater{\pgfpointanchor{mynode}{base}}

\def\pgfsetbaselinepointlater#1{\def\pgf@baseline{#1}}


% Sets the baseline at the y-coordinate of a given point, now
%
% #1 = point
%
% Sets the baseline of the picture to the y-coordinate of a given
% point.
%
% Example:
%
% \pgfsetbaselinepointnow{\pgfpoint{1cm}{2pt}}

\def\pgfsetbaselinepointnow#1{%
  \pgf@process{#1}%
  \edef\pgf@setter@baseline{\noexpand\pgfpoint{\the\pgf@x}{\the\pgf@y}}%
  \pgfsetbaselinepointlater{\pgf@setter@baseline}%
}

\def\pgf@default@text{default}%

% Sets the baseline
%
% #1 = baseline
%
% Sets the baseline of the picture. Default is the lower border, which
% is the same as \pgf@picminy
%
% Example:
%
% \pgfsetbaseline{1cm+2pt}
% \pgfsetbaseline{default}% resets to default value

\def\pgfsetbaseline#1{%
  \def\pgf@temp{#1}%
  \ifx\pgf@temp\pgf@default@text
    \pgfsetbaseline{\pgf@picminy}%
  \else
    \pgfsetbaselinepointlater{\pgfpoint{0pt}{#1}}%
  \fi
}
\pgfsetbaseline{\pgf@picminy}

% controls how the image externalization implements trim:
\newif\ifpgf@trim@lowlevel
\pgfkeys{
  /pgf/trim lowlevel/.is if=pgf@trim@lowlevel,
  /pgf/trim lowlevel/.default=true,
}

% Same as the y-baseline for horizontal alignment.
% The effect is different, though: it is some kind of trimming which
% leaves the bounding box intact.
\def\pgfsettrimleftpointlater#1{\def\pgf@trimleft{#1}}
\def\pgfsettrimleftpointnow#1{%
  \pgf@process{#1}%
  \edef\pgf@setter@baseline{\noexpand\pgfpoint{\the\pgf@x}{\the\pgf@y}}%
  \pgfsettrimleftpointlater{\pgf@setter@baseline}%
}
% \pgfsettrimleft{<x coord>}
% or
% \pgfsettrimleft{default}
\def\pgfsettrimleft#1{%
  \def\pgf@temp{#1}%
  \ifx\pgf@temp\pgf@default@text
    \pgfsettrimleft{\pgf@picminx}
  \else
    \pgfsettrimleftpointlater{\pgfpoint{#1}{0pt}}%
  \fi
}
\pgfsettrimleft{\pgf@picminx}

\def\pgfsettrimrightpointlater#1{\def\pgf@trimright{#1}}
\def\pgfsettrimrightpointnow#1{%
  \pgf@process{#1}%
  \edef\pgf@setter@baseline{\noexpand\pgfpoint{\the\pgf@x}{\the\pgf@y}}%
  \pgfsettrimrightpointlater{\pgf@setter@baseline}%
}
% \pgfsettrimright{<x coord>}
% or
% \pgfsettrimright{default}
\def\pgfsettrimright#1{%
  \def\pgf@temp{#1}%
  \ifx\pgf@temp\pgf@default@text
    \pgfsettrimright{\pgf@picmaxx}%
  \else
    \pgfsettrimrightpointlater{\pgfpoint{#1}{0pt}}%
  \fi
}
\pgfsettrimright{\pgf@picmaxx}


% Interrupt path
%
% Description:
%
% The environment can be used to insert some drawing commands while
% constructing a path. The drawing commands inside the environment
% will not interfere with the path being constructed ``outside.''
% However, you must ward against graphic state changes using a scope. 
%
% Example: Draw two parallel lines
%
% \pgfmoveto{\pgfpoint{0cm}{0cm}}
% \begin{pgfinterruptpath}
%   \pgfmoveto{\pgfpoint{1cm}{0cm}}
%   \pgfmoveto{\pgfpoint{1cm}{1cm}}
%   \pgfusepath{stroke}
% \end{pgfinterruptpath}
% \pgflineto{\pgfpoint{0cm}{1cm}}
% \pgfusepath{stroke}

\def\pgfinterruptpath
{%
  \begingroup%
  % save all sorts of things...
  \edef\pgf@interrupt@savex{\the\pgf@path@lastx}%
  \edef\pgf@interrupt@savey{\the\pgf@path@lasty}%
  \pgf@getpathsizes\pgf@interrupt@pathsizes%
  \pgfsyssoftpath@getcurrentpath\pgf@interrupt@path%
  \pgfsyssoftpath@setcurrentpath\pgfutil@empty%
  \edef\pgfscope@linewidth{\the\pgflinewidth}%
  \let\pgf@interrupt@lastmoveto=\pgfsyssoftpath@lastmoveto%
  \let\pgf@interrupt@nlt@infos=\pgf@nlt@infos%
  \begingroup%
}
\def\endpgfinterruptpath
{%
  \endgroup%
  \global\pgflinewidth=\pgfscope@linewidth%
  \pgfsyssoftpath@setcurrentpath\pgf@interrupt@path%
  \pgf@setpathsizes\pgf@interrupt@pathsizes%
  \global\pgf@path@lastx\pgf@interrupt@savex%
  \global\pgf@path@lasty\pgf@interrupt@savey%
  \global\let\pgfsyssoftpath@lastmoveto\pgf@interrupt@lastmoveto%
  \global\let\pgf@nlt@infos=\pgf@interrupt@nlt@infos%
  \endgroup%
}



% Interrupt bounding box
%
% Description:
%
% The environment can be used to temporarily setup a new bounding box
% computation. The bounding box will be made empty at the beginning of
% the environment and will be reset to its old value after the
% environment. 
%
% Example: 
%
% \begin{pgfinterruptboundinbox}
%   \pgfmoveto{\pgfpoint{1cm}{0cm}}
%   \pgfmoveto{\pgfpoint{1cm}{1cm}}
%   \pgfusepath{stroke}
% \end{pgfinterruptboundinbox}

\def\pgfinterruptboundingbox
{%
  \begingroup%
    \edef\pgf@interrupt@savemaxx{\the\pgf@picmaxx}%
    \edef\pgf@interrupt@saveminx{\the\pgf@picminx}%
    \edef\pgf@interrupt@savemaxy{\the\pgf@picmaxy}%
    \edef\pgf@interrupt@saveminy{\the\pgf@picminy}%
    \pgf@picmaxx=-16000pt\relax%
    \pgf@picminx=16000pt\relax%
    \pgf@picmaxy=-16000pt\relax%
    \pgf@picminy=16000pt\relax%
    \pgf@size@hookedfalse%
    \let\pgf@path@size@hook=\pgfutil@empty%
}
\def\endpgfinterruptboundingbox
{%
    \global\pgf@picmaxx=\pgf@interrupt@savemaxx%
    \global\pgf@picmaxy=\pgf@interrupt@savemaxy%
    \global\pgf@picminx=\pgf@interrupt@saveminx%
    \global\pgf@picminy=\pgf@interrupt@saveminy%
  \endgroup%
}





% Interrupts a picture
%
% Description:
%
% This environment interrupts a picture and temporarily returns to
% normal TeX mode. All sorts of things are saved and restored by this
% environment.
%
% WARNING: Using this environment in conjuction with low level
% transformations can *strongly* upset the typesetting. Typically, the
% contents of this environment should have size/height/depth 0pt in
% the end.
%
% WARNING: This environment should only be used inside typesetting a
% box and this box must in turn be inserted using \pgfqbox.
%
% Example: Draw two parallel lines
%
% \pgfmoveto{\pgfpoint{0cm}{0cm}}
% \setbox\mybox=\hbox{
%    \begin{pgfinterruptpicture}
%      This is normal text.
%      \begin{pgfpicture} % a subpicture
%        \pgfmoveto{\pgfpoint{1cm}{0cm}}
%        \pgfmoveto{\pgfpoint{1cm}{1cm}}
%        \pgfusepath{stroke}
%      \end{pgfpicture}
%      More text.
%    \end{pgfinterruptpicture}
%  }
% \ht\mybox=0pt
% \wd\mybox=0pt
% \dp\mybox=0pt
% \pgfqbox{\mybox}%
% \pgfpathlineto{\pgfpoint{0cm}{1cm}}
% \pgfusepath{stroke}

\def\pgfinterruptpicture
{%
  \begingroup%
    \pgfinterruptboundingbox%
      \pgftransformreset%
      \pgfinterruptpath%
        \ifx\pgf@selectfontorig\@undefined%
        \else%
          \let\setlength\pgf@setlengthorig%
          \let\addtolength\pgf@addtolengthorig%
          \let\selectfont\pgf@selectfontorig%
        \fi%
        \pgfutil@selectfont%
        \pgfpicturefalse%
        \let\pgf@positionnodelater@macro\relax%
        \pgf@savelayers%
}
\def\endpgfinterruptpicture
{%
        \pgf@restorelayers%
      \endpgfinterruptpath%
    \endpgfinterruptboundingbox%
  \endgroup%
}

\let\pgf@savelayers=\relax
\let\pgf@restorelayers=\relax

\endinput
